\documentclass[a4paper,fontsize=11pt,parskip=off,open=any,bibtotoc,headings=normal,appendixrefix=true,listof=totoc]{scrbook}

% \usepackage[automark]{scrpage2}
% \pagestyle{scrheadings}

%\usepackage{fontspec}
%\setmainfont{Arial}
%\setmainfont[Ligatures=TeX]{Helvetica}
%\usepackage{helvet}

\usepackage[top=2.5cm,bottom=2cm,left=2.5cm,right=3cm,footskip=1cm]{geometry}
\usepackage[ngerman]{babel}
%hyph-utf8 should automatically have been loaded

%\usepackage{microtype}
%\usepackage{lipsum}

%\usepackage{xcolor}
%\usepackage{tocstyle}

\iffalse
%Testing for PDF latex, does not work

%\usepackage[scaled]{uarial}
\usepackage[scaled]{helvet}
\renewcommand*\familydefault{\sfdefault} %% Only if the base font of the 
% document is to be sans serif
\usepackage[T1]{fontenc}
\usepackage[utf8]{inputenc}
\usepackage{microtype}

\else
%lualatex

%x\usepackage{luaotfload}
%\usepackage[EU2]{fontenc}

%\pdfprotrudechars % protrusion without affecting line breaks
%\pdfadjustspacing % expansion without affecting line breaks
\pdfprotrudechars=2 % protrusion with affecting line breaks
\pdfadjustspacing=2 % expansion with affecting line breaks

\usepackage{fontspec}
%\newfontfamily\ArialRounded{Arial Rounded MT Bold}
%\addtokomafont{disposition}{\ArialRounded\normalsize}

%\newfontfamily\ArialRounded{Arial Rounded MT Bold}
\addtokomafont{disposition}{\normalfont\bfseries}
%\addtokomafont{chapter}{\normalsize}
%\addtokomafont{section}{\normalsize}
%\addtokomafont{subsection}{\normalsize}
%\addtokomafont{pagehead}{\normalfont\scshape}

\newfontfeature{Microtype}{protrusion=off;expansion=default;}

%\setmainfont[RawFeature={protrusion=default;expansion=default;},Ligatures=TeX]{Linux Libertine O}
\setmainfont[Microtype,Ligatures=TeX,Renderer=Basic]{Arial}
%\setmainfont[Ligatures=TeX]{Arial}
\fi

\usepackage{csquotes}
%\usepackage[style=french,french=guillemets*]{csquotes} % zu viel Platz
%Tipp von csquotes.pdf, page 28 (Section 10.8)
%\defineshorthand{"`}{\openautoquote}
%\defineshorthand{"'}{\closeautoquote}

\usepackage[
  maxnames=99,
  maxcitenames=3,
  bibencoding=auto,
  backend=biber,
  sortlocale=de,
  style=authoryear-comp,
  sorting=nyt,
  backref=true,
  backrefstyle=two,
  hyperref=true,
  dashed=false,
  mincrossrefs=1
]{biblatex}

\DeclareFieldFormat{postnote}{S.\,#1}
\DeclareFieldFormat{multipostnote}{S.\,#1}

%Tipp von http://meinews.niuz.biz/biblatex-t473141.html
\DeclareFieldFormat{citetitle}{#1\isdot}
\DeclareFieldFormat[article]{citetitle}{{#1\isdot}}
\DeclareFieldFormat[inbook]{citetitle}{{#1\isdot}}
\DeclareFieldFormat[incollection]{citetitle}{{#1\isdot}}
\DeclareFieldFormat[inproceedings]{citetitle}{{#1\isdot}}
\DeclareFieldFormat[patent]{citetitle}{{#1\isdot}}
\DeclareFieldFormat[thesis]{citetitle}{{#1\isdot}}
\DeclareFieldFormat[unpublished]{citetitle}{{#1\isdot}}

\DefineBibliographyStrings{ngerman}{%
  urlseen = {zuletzt geprüft am},
  backrefpage = {zitiert auf S.\,},
  backrefpages = {zitiert auf S.\,}
}

\usepackage[
  pdfauthor = {{undefined}}, % Durch "Vorname Nachname" erseten -- wichtig: doppelte Klammerung
  pdftitle = {{undefined}},  % Durch den Titel der Arbeit - ohne "\\" usw. ersetzen.
  unicode,
%  pdfpagelabels=true,
  bookmarks=true,
  bookmarksnumbered=true,
%  bookmarksopen=true,
%  bookmarksopenlevel=1,
%  colorlinks=true,
%  hyperindex=true,
%  linkcolor=black, %blue, %default: red
%  citecolor=black, %Mulberry,%DarkOrchid, %default: green
%  urlcolor=black, %blue, default: cyan
  pdfview = Fit,
  pdfpagelayout=SinglePage,
]{hyperref}

\usepackage{tabularx}

\usepackage{paralist}

\usepackage{graphics}

\usepackage{ragged2e}

%Zeilenabstand wie bei Word
\usepackage{setspace}
%\onehalfspacing
\setstretch{1.4}

\bibliography{literaturverzeichnis.bib}

\newcommand{\citep}[2][]{\parencite[vgl.][#1]{#2}}
\let\citet\cite
%\let\citet\textcite

\usepackage{xspace}
%%% Fussnoten/Endnoten ===================================================
%
%%% Doc: ftp://tug.ctan.org/pub/tex-archive/macros/latex/contrib/footmisc/footmisc.pdf
%
\usepackage[
   bottom,      % Footnotes appear always on bottom. This is necessary
                % especially when floats are used
   stable,      % Make footnotes stable in section titles
   %perpage,     % Reset on each page
   %para,       % Place footnotes side by side of in one paragraph.
   %side,       % Place footnotes in the margin
   ragged,      % Use RaggedRight
   %norule,     % suppress rule above footnotes
   multiple,    % rearrange multiple footnotes intelligent in the text.
   %symbol,     % use symbols instead of numbers
]{footmisc}

\interfootnotelinepenalty=10000 % Verhindert das Fortsetzen von
                                % Fussnoten auf der gegenüberligenden Seite

\newcommand{\ua}{u.\,a.\xspace}
\newcommand{\vgl}{vgl.\xspace}
\newcommand{\zB}{z.\,B.\xspace}
\renewcommand{\dh}{d.h.\xspace}
\usepackage{xcolor}
\usepackage{xifthen}% provides \isempty test 

\definecolor{quotemark}{gray}{0.7}
\newenvironment{fquote}[1]{%
  \newcommand{\fqauthor}{\relax}
  \ifthenelse{\isempty{#1}}
  	{}% do nothing
  	{
\noindent
\begin{minipage}{\textwidth}
\vspace{.5\baselineskip}
\renewcommand{\fqauthor}{\hfill\textsc{#1}}}
%  \vspace{1em}
  \begin{list}{}{%
      \setlength{\leftmargin}{1cm}
      \setlength{\rightmargin}{1cm}
    }
    \item[]%
    \begin{picture}(0,0)(0,0)
      \put(-15,-5){\makebox(0,0){%
	  \scalebox{4.5}{\textcolor{quotemark}{\bfseries \glqq}}}%
      }
    \end{picture}\em\ignorespaces%
}{%
  \newline%
  \makebox[0pt][l]{\hspace{13.7cm}%
  \begin{picture}(0,0)(0,0)
    \put(15,10){\makebox(0,0){%
	\scalebox{4.5}{\textcolor{quotemark}{\rm\bfseries \grqq}}}%
    }
  \end{picture}}%
  \fqauthor
  \end{list}
\vspace{0\baselineskip}
\end{minipage}
}

%\input{chapterheads}

%\usepackage[compact]{titlesec}
%\titlespacing{\chapter}{0pt}{0pt}{12pt}

\raggedbottom

\renewcommand*{\chapterheadstartvskip}{\vspace*{-\topskip}\vspace{-\baselineskip}}

%\usepackage{titlesec}


\makeatletter
% Warnen, falls die vorgefundene Originaldefinition nicht
% der erwarteten Originaldefinition entspricht:
\CheckCommand{\subsection}{\@startsection{subsection}{2}{\z@}%
  {-3.25ex\@plus -1ex \@minus -.2ex}%
  {1.5ex \@plus .2ex}%
  {\ifnum \scr@compatibility>\@nameuse{scr@v@2.96}\relax
    \setlength{\parfillskip}{\z@ plus 1fil}\fi
    \raggedsection\normalfont\sectfont\nobreak\size@subsection
  }%
}
% Definition ändern:
\renewcommand{\subsection}{\@startsection{subsection}{2}{\z@}%
  {-.5\baselineskip}%
  {.5\baselineskip}%
  {\ifnum \scr@compatibility>\@nameuse{scr@v@2.96}\relax
    \setlength{\parfillskip}{\z@ plus 1fil}\fi
    \raggedsection\normalfont\sectfont\nobreak\size@subsection
  }%
}


\renewcommand\section{\@startsection{section}{1}{\z@}%
  {-.5\baselineskip}%
  {.5\baselineskip}%
  {\ifnum \scr@compatibility>\@nameuse{scr@v@2.96}\relax
    \setlength{\parfillskip}{\z@ plus 1fil}\fi
    \raggedsection\normalfont\sectfont\nobreak\size@section}%
}

\makeatother

\renewcommand*{\chapterheadstartvskip}{\vspace{-.3cm}}
%\renewcommand*{\chapterheadstartvskip}{\vspace*{-\topskip}}
\renewcommand*{\chapterheadendvskip}{\vspace{.3cm}}
%\renewcommand*{\chapterheadendvskip}{.5\baselineskip}

\begin{document}

%Falls Abbildungen im Anhang sind und es mehr als 9 sind:
%Abstände im Inhaltsverzeichnis korrigieren
\makeatletter
\renewcommand\l@figure{\@dottedtocline{0}{0em}{3em}} 
\makeatother

\setlength{\parindent}{2em}
\renewcommand{\subsectionautorefname}{Abschnitt}
\renewcommand{\bibname}{Literaturverzeichnis}

%\titlespacing{\section}{11pt}{0em}{0em}
%\titlespacing{\subsection}{11pt}{0em}{0em}
%\titlespacing{\subsubsection}{11pt}{0em}{0em}

\titlehead{\centering\Large Hochschule Esslingen\\
Fakultät: Soziale Arbeit, Gesundheit und Pflege\\
Studiengang: Bildung und Erziehung in der Kindheit (B.\,A.)
}
\subject{Bachelorarbeit}
%\subtitle{Klein aber fein?}
\date{Esslingen, 15.\ Mai 2012}
\publishers{\raggedright 
\large
\begin{tabular}{ll}
Betreuende Professorin: &Prof.\,Dr.\,paed.\,Dipl.-Päd.\ Regine Morys\\
Zweitprüferin: &Prof.\,Dr.\,phil.\,Dipl.-Sozialpäd.\,(FH)\,Dipl.-Päd.\ Ulrike Zöller
\end{tabular}
}

\author{Vorname Nachname\vspace{-.3cm}\\{\small Matrikelnummer: 11\,22\,33 | {xy00@hs-esslingen.de}}}
\title{Mein Titel\\mit einem kleinen  Umbruch}

\lowertitleback{
\href{http://creativecommons.org/licenses/by-sa/3.0/de/}{\includegraphics[scale=.5]{CC-by-sa.png}}
%Dieses Werk bzw. Inhalt steht unter einer \href{http://creativecommons.org/licenses/by-sa/3.0/de/}{Creative Commons Namensnennung-Weitergabe unter gleichen Bedingungen 3.0 Deutschland Lizenz}.

Diese Arbeit wurde mit Hilfe von \href{http://www.komascript.de/}{\KOMAScript} und \href{http://www.luatex.org/}{LuaLaTeX} gesetzt. Die Vorlage befindet über \href{https://github.com/latextemplates/SAGP}{{https://github.com/latextemplates/SAGP}} verfügbar.
}

\maketitle

 %Darf NICHT benutzt werden, da ALLE Überschriften auch im Inhaltsverzeichnis auftauchen sollen
%\setcounter{tocdepth}{1}

\tableofcontents
\listoffigures

\chapter*{Abkürzungsverzeichnis}
\addcontentsline{toc}{chapter}{Abkürzungsverzeichnis}
%\centering

\begin{tabularx}{.8\textwidth}{@{}lX}
%\begin{tabular}{lp{.6\textwidth}}
\textbf{Abkürzung} & \textbf{Bedeutung}\\
%{\ArialRounded Abkürzung} & {\ArialRounded Bedeutung}\\
BMFSFJ & Bundesministerium für Familie, Senioren, Frauen und Jugend\\
FaFo & FamilienForschung Baden-Württemberg\\
VAMV & Verband alleinerziehender Mütter und Väter, Bundesverband e.\,V.\\
\end{tabularx}

\chapter{Einleitung}
%\setcounter{page}{1} %REMOVEME
\label{Einleitung}

Zitate: Am Ende in der Klammer (\vgl \citet[9]{Peuckert2008}).

Es Kapitel, Graphiken usw.\ werden mit \autoref{Einleitung} gemacht: 
Die Arbeit wird mit dem \autoref{zusammenfassung} \enquote{\nameref{zusammenfassung}} abgeschlossen.

Anführungszeichen mit \enquote{Text}.

\section{Abschnitt}

\subsection{Unterabschnitt}

\autoref{fig:LebendGeb} zeigt die Anzahl der Lebendgeborenen je 1\,000 Frauen im Jahr 2010.

Bei Graphiken darauf achten, dass .pdf, .jpg, .png o.ä. verwendet wird. Kein .eps o.ä.

\begin{figure}[h]
\centering
\includegraphics[width=.75\textwidth]{Lebendgeborene.pdf}
\caption[Anzahl der Lebendgeborenen je 1\,000 Frauen 2010]{Anzahl der Lebendgeborenen je 1\,000 Frauen nach Alter der Mutter 2010 (aus
Statistisches Bundesamt, 2011: Statistik der Geburten in
\citet[16]{BundesministeriumfuerFamilie2011})}
\label{fig:LebendGeb}
\end{figure}


\chapter{Zusammenfassung und Ausblick}
\label{zusammenfassung}

\minisec{Ausblick}
%nicht als \section{}, da es keine zweite section hier gibt.


%Literaturverzeichnis
{
\RaggedRight
%\singlespacing %Der Standard fordert einzeilig. Es sieht allerdings nicht gut aus.
\printbibliography
}

\noindent
Die im Literaturverzeichnis aufgeführten Sammelwerke werden indirekt durch das Zitat eines
enthaltenen Beitrags referenziert. Die Formvorschriften verlangen eine Auflistung dieser im
Literaturverzeichnis.

% \printbibheading

% \printbibliography[
% title={Direkt zitierte Literatur},
% heading=subbibliography,
% notype=proceedings,
% notkeyword=nonorig]

% \printbibliography[
% title={Sammelbände},
% heading=subbibliography,
% type=proceedings,
% notkeyword=nonorig]

% \printbibliography[
% title={Zitierte nicht-vorliegende Originalquellen},
% heading=subbibliography,
% keyword=nonorig]


% Versicherung

\clearpage
\pagestyle{empty}

\centering
\begin{minipage}{.66\textwidth}
\begin{center}
{\Huge\textbf{Erklärung}}
\end{center}
\vspace{1.5\baselineskip}


Hiermit versichere ich gemäß §\,28 der Studien- und
Prüfungsordnung der Hochschule Esslingen -- Fakultät Soziale Arbeit, Gesundheit
und Pflege, dass ich diese Bachelorarbeit selbständig verfasst und keine anderen als die angegebenen Quellen und Hilfsmittel benutzt habe. 

\vspace{3ex}
Ort, 15.\ Mai 2012

\vspace{3cm}
\hrule
\vspace{.2cm}
Vorname Nachname

\end{minipage}

\end{document}
