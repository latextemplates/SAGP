% -*- coding:utf-8 mod:LaTeX -*-

%%%%%%%%%%
% 8-UNG; VOR DRUCK ZUR ABGABE FOLGENDES BEACHTEN:
% - Einseitig oder Zweiseitiger Druck? (-> Entsprechende Option oben.)
% - Nach TBD in der Datei suchen und Probleme beheben.
% - Abkürzungen korrekt setzen (z. B. "\zB" statt "z.B." oder ähnlichem).
% - Korrekte Anführungszeichen setzten (z. B. "`So ist's gut"' für deutsche Anführungszeichen).
% - Rechtschreibkorrektur über den Text lassen.
% - Skript zur Erkennung von Wortverdopplungen drüber lassen.
% - Silbentrennung.
% - Titel, Namen, Matrikelnummer, E-Mail, Datum, Betreuer\_Innen (und sonstige Angaben) auf dem Titelblatt überprüfen.
% - Formulierung der Erklärung überprüfen und 'Ort' und 'Datum' überprüfen und 'Vorname Nachname' anpassen.
% - Copyright und \LaTeX-Hinweis gewollt?
% - Hinweis in Bibliographie gewollt?


%%%%%%%%%%
\documentclass[a4paper,oneside,fontsize=11pt,headsepline,parskip=off,open=any,headings=small,listof=totoc,bibliography=totoc]{scrbook}


%%%%%%%%%%
% Ränder und auf deutsch.
\usepackage[a4paper,left=2.5cm,right=3.3cm,top=3.3cm,bottom=3.0cm,footskip=1cm,twoside=false]{geometry}
% Alternativ: top=2.5cm,bottom=2cm,left=2.5cm,right=3cm,footskip=1cm
\usepackage[ngerman]{babel}


%%%%%%%%%%
% Zeilenabstand 1,5-zeilig
\usepackage{setspace}
% ACHTUNG: \onehalfspacing ist zu klein gesetzt, wenn man es gegen
% Word misst! \setstretch{1.4} kommt eher hin.
%\onehalfspacing
\setstretch{1.4}
\raggedbottom


%%%%%%%%%%
% pdflatex / lualatex?
\iftrue % pdflatex

% TBD: Scaled oder nicht scaled, das ist hier die Frage.
% Source: http://www.tug.dk/FontCatalogue/helvetica/
%\usepackage[scaled]{helvet}
\usepackage{helvet}
%\usepackage[scaled]{uarial} % Alternative...
 %% Only if the base font of the document is to be sans serif
\renewcommand*{\familydefault}{\sfdefault}

\usepackage[T1]{fontenc}
\usepackage[utf8]{inputenc}
\usepackage{microtype}

\else % lualatex
%\usepackage{luaotfload}
%\usepackage[EU2]{fontenc}

%\pdfprotrudechars % protrusion without affecting line breaks
%\pdfadjustspacing % expansion without affecting line breaks
\pdfprotrudechars=2 % protrusion with affecting line breaks
\pdfadjustspacing=2 % expansion with affecting line breaks

\usepackage{fontspec}
%\newfontfamily\ArialRounded{Arial Rounded MT Bold}
%\addtokomafont{disposition}{\ArialRounded\normalsize}

%\newfontfamily\ArialRounded{Arial Rounded MT Bold}
\addtokomafont{disposition}{\normalfont\bfseries}
%\addtokomafont{chapter}{\normalsize}
%\addtokomafont{section}{\normalsize}
%\addtokomafont{subsection}{\normalsize}
%\addtokomafont{pagehead}{\normalfont\scshape}

\newfontfeature{Microtype}{protrusion=off;expansion=default;}

%\setmainfont[RawFeature={protrusion=default;expansion=default;},Ligatures=TeX]{Linux Libertine O}
\setmainfont[Microtype,Ligatures=TeX,Renderer=Basic]{Arial}
%\setmainfont[Ligatures=TeX]{Arial}
\fi
% END: pdflatex / lualatex?
%%%%%%%%%%


% TBD: Alle notwendig?
\usepackage{fixltx2e,graphicx,ragged2e,xspace,xcolor,url,enumerate}
% Herausgenommen: paralist,tabularx
% \usepackage{xifthen} % Stellt \isempty zur Verfuegung


%%%%%%%%%%
% Fortgeschrittene Zitate.
\usepackage{csquotes}
%\usepackage[style=french,french=guillemets*]{csquotes} % zu viel Platz
% % Tipp von csquotes.pdf, Seite 28, Abschnitt 10.8:
% \defineshorthand{"`}{\openautoquote}
% \defineshorthand{"'}{\closeautoquote}


% %%%%%%%%%%
% % Alternative layouts. DEAKTIVIERT.
% \usepackage[automark]{scrpage2}
% \pagestyle{scrheadings}
% \usepackage{tocstyle}


%%%%%%%%%%
% Bibliographie aufsetzen.
\usepackage[
  backend=bibtex,
  % Alternativ dazu: backend=biber,
  maxnames=99,
  maxcitenames=3,
  bibencoding=auto,
  sortlocale=de,
  style=authoryear-comp,
  sorting=nyt,
  backref=true,
  backrefstyle=two,
  hyperref=true,
  dashed=false,
  mincrossrefs=1
]{biblatex}

\DeclareFieldFormat{postnote}{S.\,#1}
\DeclareFieldFormat{multipostnote}{S.\,#1}

% Tipp von http://meinews.niuz.biz/biblatex-t473141.html
\DeclareFieldFormat{citetitle}{#1\isdot}
\DeclareFieldFormat[article]{citetitle}{{#1\isdot}}
\DeclareFieldFormat[inbook]{citetitle}{{#1\isdot}}
\DeclareFieldFormat[incollection]{citetitle}{{#1\isdot}}
\DeclareFieldFormat[inproceedings]{citetitle}{{#1\isdot}}
\DeclareFieldFormat[patent]{citetitle}{{#1\isdot}}
\DeclareFieldFormat[thesis]{citetitle}{{#1\isdot}}
\DeclareFieldFormat[unpublished]{citetitle}{{#1\isdot}}

\DefineBibliographyStrings{ngerman}{%
  urlseen = {zuletzt geprüft am},
  backrefpage = {zitiert auf S.\,},
  backrefpages = {zitiert auf S.\,}
}

\bibliography{literaturverzeichnis.bib}

\newcommand{\citep}[2][]{\parencite[vgl.][#1]{#2}}
\let\citet\cite
%\let\citet\textcite


%%%%%%%%%%
% Fussnoten/Endnoten
% Doc: ftp://tug.ctan.org/pub/tex-archive/macros/latex/contrib/footmisc/footmisc.pdf
\usepackage[
   bottom,      % Footnotes appear always on bottom. This is necessary
                % especially when floats are used
   stable,      % Make footnotes stable in section titles
   %perpage,     % Reset on each page
   %para,       % Place footnotes side by side of in one paragraph.
   %side,       % Place footnotes in the margin
   ragged,      % Use RaggedRight
   %norule,     % suppress rule above footnotes
   multiple,    % rearrange multiple footnotes intelligent in the text.
   %symbol,     % use symbols instead of numbers
]{footmisc}

\interfootnotelinepenalty=10000 % Verhindert das Fortsetzen von
                                % Fussnoten auf der gegenüberligenden Seite


%%%%%%%%%%
% Hypertext Referenzen.
\usepackage[
  unicode,
  bookmarks=true,
  bookmarksnumbered=true,
%  bookmarksopen=true,
%  bookmarksopenlevel=1,
%  colorlinks=true,
%  pdfpagelabels=true,
%  hyperindex=true,
%  linkcolor=black, %blue, %default: red
%  citecolor=black, %Mulberry,%DarkOrchid, %default: green
%  urlcolor=black, %blue, default: cyan
  pdfview=Fit,
  pdfpagelayout=SinglePage,
]{hyperref}


% TBD: Was ist das alles?
\definecolor{quotemark}{gray}{0.7}
\newenvironment{fquote}[1]{%
  \newcommand{\fqauthor}{\relax}
  \ifthenelse{\isempty{#1}}
  	{}% do nothing
  	{
\noindent
\begin{minipage}{\textwidth}
\vspace{.5\baselineskip}
\renewcommand{\fqauthor}{\hfill\textsc{#1}}}
%  \vspace{1em}
  \begin{list}{}{%
      \setlength{\leftmargin}{1cm}
      \setlength{\rightmargin}{1cm}
    }
    \item[]%
    \begin{picture}(0,0)(0,0)
      \put(-15,-5){\makebox(0,0){%
	  \scalebox{4.5}{\textcolor{quotemark}{\bfseries \glqq}}}%
      }
    \end{picture}\em\ignorespaces%
}{%
  \newline%
  \makebox[0pt][l]{\hspace{13.7cm}%
  \begin{picture}(0,0)(0,0)
    \put(15,10){\makebox(0,0){%
	\scalebox{4.5}{\textcolor{quotemark}{\rm\bfseries \grqq}}}%
    }
  \end{picture}}%
  \fqauthor
  \end{list}
\vspace{0\baselineskip}
\end{minipage}
}

%\input{chapterheads}

%\usepackage[compact]{titlesec}
%\titlespacing{\chapter}{0pt}{0pt}{12pt}


\renewcommand*{\chapterheadstartvskip}{\vspace*{-\topskip}\vspace{-\baselineskip}}
\renewcommand*{\chapterheadendvskip}{\vspace{.3cm}}
%\renewcommand*{\chapterheadstartvskip}{\vspace{-.3cm}}
%\renewcommand*{\chapterheadstartvskip}{\vspace*{-\topskip}}
%\renewcommand*{\chapterheadendvskip}{.5\baselineskip}

% TBD: Was bewirkt das Folgende?
\iffalse
%\usepackage{titlesec}

\makeatletter
% Warnen, falls die vorgefundene Originaldefinition nicht
% der erwarteten Originaldefinition entspricht:
\CheckCommand{\subsection}{\@startsection{subsection}{2}{\z@}%
  {-3.25ex\@plus -1ex \@minus -.2ex}%
  {1.5ex \@plus .2ex}%
  {\ifnum \scr@compatibility>\@nameuse{scr@v@2.96}\relax
    \setlength{\parfillskip}{\z@ plus 1fil}\fi
    \raggedsection\normalfont\sectfont\nobreak\size@subsection
  }%
}
% Definition ändern:
\renewcommand{\subsection}{\@startsection{subsection}{2}{\z@}%
  {-.5\baselineskip}%
  {.5\baselineskip}%
  {\ifnum \scr@compatibility>\@nameuse{scr@v@2.96}\relax
    \setlength{\parfillskip}{\z@ plus 1fil}\fi
    \raggedsection\normalfont\sectfont\nobreak\size@subsection
  }%
}

\renewcommand\section{\@startsection{section}{1}{\z@}%
  {-.5\baselineskip}%
  {.5\baselineskip}%
  {\ifnum \scr@compatibility>\@nameuse{scr@v@2.96}\relax
    \setlength{\parfillskip}{\z@ plus 1fil}\fi
    \raggedsection\normalfont\sectfont\nobreak\size@section}%
}
\makeatother\fi

%%%%%%%%%%
% \date und \author umbiegen, damit dieses in \OrtDatumWieImTitel
% bzw. \AutorWieImTitel zwischengespeichert wird (scrbook schmeisst
% die Werte bei \maketitle weg).
\makeatletter%
\let\tempD@teTaqcRdbh\date% altes \date zwischenspeichern.
\let\tempAuthorTaqcRdbh\author% altes \date zwischenspeichern.
\let\tempmaketitleTaqcRdbh\maketitle% altes \maketitle zwischenspeichern.
\newcommand{\OrtDatumWieImTitel}{Ort und Datum unbekannt (\textbackslash date aufrufen zum Setzen).}
\newcommand{\AutorWieImTitel}{Autor unbekannt (\textbackslash author aufrufen zum Setzen).}
\newcommand{\MatrikelNrWieImTitel}{Matrikelnr.\ unbekannt (\textbackslash matrikelno aufrufen zum Setzen).}
\newcommand{\EMailWieImTitel}{E-Mail Adresse unbekannt (\textbackslash email aufrufen zum Setzen).}
\renewcommand{\date}[1]{\renewcommand{\OrtDatumWieImTitel}{#1}}%
\renewcommand{\author}[1]{\renewcommand{\AutorWieImTitel}{#1}}%
\newcommand{\matrikelno}[1]{\renewcommand{\MatrikelNrWieImTitel}{#1}}%
\newcommand{\email}[1]{\renewcommand{\EMailWieImTitel}{#1}}%
\renewcommand{\maketitle}{%
  \subject{Bachelorarbeit}%
  \tempD@teTaqcRdbh{\OrtDatumWieImTitel}%
  \tempAuthorTaqcRdbh{{\AutorWieImTitel}%
    \texorpdfstring{\vspace{-.1cm}\\{\small%
        Matrikelnummer: {\MatrikelNrWieImTitel} | {\texttt{\EMailWieImTitel}}}}{}}%
  \tempmaketitleTaqcRdbh}
\makeatother


%%%%%%%%%%
% Um "To Be Dones" zu markieren. Benutzung via
% \TBD[Text]
% Wichtig: In eckigen Klammern.  Gibt einen roten Text am Rand aus und
% eine Meldung in die log-Datei.  Wird kein Text angegeben, wird eine
% Standardmeldung ausgegeben.
% TBD: Vor dem Druck auskommentieren, es sollten dann keine Fehler
% auftreten.
%
% TBD für das TBD: Warum werden zusätzliche Leerzeichen im Fließtext
% eingefügt?
\newcommand{\TBD}[1][\relax]{\ifx#1\relax%
  \marginpar{\raggedright\footnotesize\singlespacing\color{red}{To Be Done.}}%
  \typeout{warning: Unresolved to-be-done}\else%
  \marginpar{\raggedright\footnotesize\singlespacing\color{red}#1}%
  \typeout{warning: Unresolved to-be-done: #1}%
  \fi\xspace}


%%%%%%%%%%
% Abkürzungen.
\newcommand{\ua}{u.\,a.\xspace}
\newcommand{\vgl}{vgl.\xspace}
\newcommand{\zB}{z.\,B.\xspace}
\renewcommand{\dh}{d.\,h.\xspace}


%%%%%%%%%%
% Anfang Inhalt.
\begin{document}

\setlength{\parindent}{2em}
\renewcommand{\subsectionautorefname}{Abschnitt}
\renewcommand{\bibname}{Literaturverzeichnis}

%\titlespacing{\section}{11pt}{0em}{0em}
%\titlespacing{\subsection}{11pt}{0em}{0em}
%\titlespacing{\subsubsection}{11pt}{0em}{0em}


%%%%%%%%%%
% Titelseite.
%\author{Vorname Nachname}%
%\matrikelno{11\,22\,33}%
%\email{xy00@hs-esslingen.de}%
\title{Mein Titel\texorpdfstring{\\}{ }mit einem kleinen Umbruch}%
% \subtitle{Klein aber oho?}%
%\date{Esslingen, \today}%
\publishers{\raggedright\large\begin{tabular}{ll}
    Betreuende Professorin: &Prof.\,Dr.\,paed.\,Dipl.-Päd.\ Regine Morys\\
    Zweitprüferin: &Prof.\,Dr.\,phil.\,Dipl.-Sozialpäd.\,(FH)\,Dipl.-Päd.\ Ulrike Zöller
  \end{tabular}}%
\titlehead{\centering\Large Hochschule Esslingen\\%
  Fakultät für Soziale Arbeit, Gesundheit und Pflege\\%
  Studiengang Bildung und Erziehung in der Kindheit (B.\,A.)}%


% Titel und Autoren in PDF Metadaten setzen.
\makeatletter\hypersetup{pdftitle={\@title}}\hypersetup{pdfauthor=\@author}\makeatother

\maketitle\cleardoublepage



%%%%%%%%%%
% Inhaltsverzeichnis
% % Folgendes darf NICHT benutzt werden, da ALLE Überschriften auch im
% % Inhaltsverzeichnis auftauchen sollen.
% \setcounter{tocdepth}{1}

% Falls Abbildungen im Anhang sind und es mehr als 9 sind:
% Abstände im Inhaltsverzeichnis korrigieren
\makeatletter\renewcommand\l@figure{\@dottedtocline{0}{0em}{3em}}\makeatother

\tableofcontents\cleardoublepage


%%%%%%%%%%
\chapter{Einleitung}
%\setcounter{page}{1} %REMOVEME
\label{Einleitung}
Dies ist ein LaTeX-Dokument, das den Formvorschriften der Fakultät SAGP von Anfang 2012 folgt.
Insbesondere wird die Schriftart Arial mit dem korrekten Zeilenabstand verwendet.


\section{Abschnitt}
Zitate: Am Ende in der Klammer (\vgl \citet[9]{Peuckert2008}).

Es Kapitel, Graphiken usw.\ werden mit \autoref{Einleitung} gemacht: 
Die Arbeit wird mit dem \autoref{zusammenfassung} \enquote{\nameref{zusammenfassung}} abgeschlossen.

Anführungszeichen mit \enquote{Text}.


\subsection{Unterabschnitt}

\autoref{fig:LebendGeb} zeigt die Anzahl der Lebendgeborenen je 1\,000 Frauen im Jahr 2010.

Bei Graphiken darauf achten, dass .pdf, .jpg, .png o.ä. verwendet wird. Kein .eps o.ä.

\begin{figure}[h]
\centering
\includegraphics[width=.75\textwidth]{Lebendgeborene.pdf}
\caption[Anzahl der Lebendgeborenen je 1\,000 Frauen 2010]{Anzahl der Lebendgeborenen je 1\,000 Frauen nach Alter der Mutter 2010 (aus
Statistisches Bundesamt, 2011: Statistik der Geburten in
\citet[16]{BundesministeriumfuerFamilie2011})}
\label{fig:LebendGeb}
\end{figure}


\chapter{Zusammenfassung und Ausblick}\label{zusammenfassung}

\minisec{Ausblick}
%nicht als \section{}, da es keine zweite section hier gibt.


%%%%%%%%%%
% Abkürzungsverzeichnis
\chapter*{Abkürzungsverzeichnis}
\addcontentsline{toc}{chapter}{Abkürzungsverzeichnis}
% \centering

\begin{tabular}{ll}
  % \begin{tabular}{lp{.6\textwidth}}
  \textbf{Abkürzung} & \textbf{Bedeutung}\\
  % {\ArialRounded Abkürzung} & {\ArialRounded Bedeutung}\\
  BMFSFJ & Bundesministerium für Familie, Senioren, Frauen und Jugend\\
  FaFo & FamilienForschung Baden-Württemberg\\
  VAMV & Verband alleinerziehender Mütter und Väter, Bundesverband e.\,V.\\
\end{tabular}


%%%%%%%%%%
% Abbildungsverzeichnis
\listoffigures


%%%%%%%%%%
% Literaturverzeichnis
\cleardoublepage
{
  \RaggedRight
  % \singlespacing %Der Standard fordert einzeilig. Das sieht allerdings nicht gut aus.
  \printbibliography
}

\noindent%
\emph{Hinweis:} %
Die im Literaturverzeichnis aufgeführten Sammelwerke werden indirekt
durch das Zitat eines enthaltenen Beitrags referenziert. Die
Formvorschriften verlangen eine Auflistung dieser im
Literaturverzeichnis.

% \printbibheading

% \printbibliography[
% title={Direkt zitierte Literatur},
% heading=subbibliography,
% notype=proceedings,
% notkeyword=nonorig]

% \printbibliography[
% title={Sammelbände},
% heading=subbibliography,
% type=proceedings,
% notkeyword=nonorig]

% \printbibliography[
% title={Zitierte nicht-vorliegende Originalquellen},
% heading=subbibliography,
% keyword=nonorig]


%%%%%%%%%%
% Versicherung
\cleardoublepage
\pagestyle{empty}
\chapter*{Eidesstattliche Erklärung}
\addcontentsline{toc}{chapter}{Eidesstattliche Erklärung}

\begin{center}
  \begin{minipage}{\textwidth}
%     \begin{center}
%       {\Huge\textbf{Eidesstattliche Erklärung}}
%     \end{center}
%     \vspace{1.5\baselineskip}
%   
    Hiermit versichere ich gemäß §\,28 der Studien- und
    Prüfungsordnung der Hochschule Esslingen -- Fakultät Soziale
    Arbeit, Gesundheit und Pflege, dass ich diese Bachelorarbeit
    selbständig verfasst und keine anderen als die angegebenen Quellen
    und Hilfsmittel benutzt habe.

      \vspace{3ex}%
    \OrtDatumWieImTitel % Wie im Titelblatt
    % Ort, \today % Frei angepassbar.
  
  
    \vspace{2cm}\hrule\vspace{.2cm}%
    \AutorWieImTitel % Wie im Titelblatt
    % Vorname Nachname % Frei angepassbar.
  \end{minipage}
\end{center}

\null\vfill%
\noindent%
\href{http://creativecommons.org/licenses/by-sa/3.0/de/}{\includegraphics[scale=.5]{by-sa}}\\
Dieses Werk bzw.\ Inhalt steht unter der Lizenz
\href{http://creativecommons.org/licenses/by-sa/3.0/de/}{Creative
  Commons Namensnennung-Weitergabe unter gleichen Bedingungen 3.0
  Deutschland Lizenz}.  Nähere Informationen sind unter
\url{http://creativecommons.org/licenses/by-sa/3.0/de/} verfügbar.

\medskip\noindent%
Diese Arbeit wurde mit Hilfe von
\href{http://www.komascript.de/}{\KOMAScript}
und %\href{http://www.luatex.org/}{LuaLaTeX}
\LaTeX\ %
gesetzt.
%
Eine Vorlage ist unter
\href{https://github.com/latextemplates/SAGP}{{https://github.com/latextemplates/SAGP}}
verfügbar.

\end{document}
